\documentclass[a4paper,12pt]{extarticle}
\usepackage[utf8]{inputenc}
%\usepackage[hungarian]{babel}
\usepackage[table,xcdraw]{xcolor}
\usepackage{physics}
\usepackage{fancyhdr}
\usepackage{geometry}
\usepackage{natbib}
\usepackage{graphicx}
\usepackage{float}
\usepackage{wrapfig}
%\usepackage{caption}
\usepackage[justification=centering]{caption}
\usepackage{subcaption}
\usepackage{gensymb}
\usepackage[export]{adjustbox}
\usepackage{hyperref}
\geometry{margin = 20mm}
\usepackage{mathtools}
\usepackage{amsmath}
\usepackage{indentfirst}
\usepackage{xurl}
\usepackage{t1enc}
\usepackage{setspace}


\title{Geant4 report 2}
\author{Bendegúz Borkovits T7UR9P}
\date{March 2022}


\begin{document}
\onehalfspacing

\maketitle

\begin{center}

\includegraphics[scale=0.3]{elte_cimer_szines.jpg}

\vspace{2 cm}
Scientific Data Analytics and Modeling specialization

Scientific Modeling Computer Laboratory

Supervisor: Ákos Horváth

\vspace{1 cm}

\end{center}

\section{Introduction}
This report contains the continuation of my Geant4 project I have been working on. Last time, I have managed to set up an environment for the software. Then, I installed Geant4 and tested it using one of its more basic built-in example scripts. During the past two weeks, I have started to learn how to create my own Geant4 scripts. This process included learning about the predefined settings of the software that are contained within its numerous built-in functions and libraries and about the different steps needed to be defined for the project to work successfully. 

\section{Creating a Geant4 project from scratch}
With the help of tutorials, I have created a Geant4 program. For said program to work, one needs to complete the following steps. First, create a run manager, that contains the main function of the program and the execution commands. Then, define the construction of the detector. After that, create an action manager that will do most of the computing. Then, define the particle generator and finally include a suitable physics list.

The cleanest way to create the program is to divide the respective parts into different files. This way one can have a rather clean look at the stricture of the project and make debugging much easier. Each part, with the exception of the run manager, is contained within a source and a header file. Header files are needed to make inclusions between files possible and to declare which built-in package one will use. The detector construction part contains the defining features of the mother volume, the detector geometry and the detector material with its properties such as the refractive index. One can define the material by including certain libraries that contain the definition of the needed molecules. Then, by declaring them as variables, one is able to construct more complex molecules to form the detector material. The particles that will be detected are generated by a predefined way. The particle generator part of the program should contain the type, position within the volume, direction of momentum of the particle and finally the type of the particle generator. Then, the physics list includes the different physical laws one needs for simulating the particles, the detector and the environment. Inclusion of certain physics lists allows the existence of different particles like optical photons. Finally, the action manager will run the particle generating and most of the computing.

For my first project, I wanted to simulate a proton reaching the detector, while also displaying the optical photons in the process. For this purpose, I constructed an aerogel detector from silicon, water and carbon molecules. It should be noted that silicon and water were also needed to be built up from their respective elements. For the mother volume, a simple box was defined with each of its edges being 1 m long. Compared to this, the detector needed to be smaller, therefore it was defined as a box with 0.8 m, 0.8 m and 0.02 m edge sizes. The particle generator was included as a simple particle gun that fires a proton in Z-direction with 100 GeV momentum. As for the physical properties, I have included the electromagnetic interaction and optical photons to display the additionally wanted particles. However, the photons did not show up. After some investigations, I have concluded that the cause of this error might be a compatibility issue of the built-in function with the current version of Geant4. I have deduced this from the fact that the proton, the detector and the mother volume did show up during the visualization.

\section{Discussion}
Apart from that one problem with the photons, the simulation ran successfully. However, the day I intended to save the figure that contained the results, my virtual machine crashed. As a result, VirtualBox forgot the ISO file of my Ubuntu machine. Due to this, the machine deleted itself and had to be reinstalled. Then, it refused to be installed and crashed. Thankfully, I have already saved the scripts of the simulation to my real machine. To amend this problem, I asked a PhD student to help me install an Ubuntu on my laptop to dual boot. However, due to an administrator level BIOS issue, we were not successful. As a solution, my supervisor has offered to allow me access to a machine from the university to continue my work with Geant4. It is for these reasons I was not able to show the figure in question in my current report.

\section{References}
\begin{enumerate}
    \item Geant4: \url{https://geant4.web.cern.ch/support/download}
    \item Tutorial: \url{https://www.youtube.com/watch?v=Lxb4WZyKeCE&list=PLLybgCU6QCGWgzNYOV0SKen9vqg4KXeVL} 
\end{enumerate}





\end{document}
